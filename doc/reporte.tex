\documentclass[11pt,letterpaper]{article}
\usepackage[margin=1.5cm,includefoot]{geometry}
\usepackage[spanish]{babel}

\usepackage[utf8]{inputenc}

\usepackage[colorlinks = true,
            linkcolor = blue,
            urlcolor  = blue,
            citecolor = blue,
            anchorcolor = blue]{hyperref}

\usepackage{listings}
\usepackage{xcolor}
\usepackage{amssymb}
\usepackage{amsmath}
\usepackage{listings}
\usepackage{graphicx}
\usepackage{multicol}
%%\graphicspath{ {img/} }

\title{Programación Declarativa \\ {\Reporte Proyecto}\\
  {\it Automatás Traductores. Ejemplos, traducción y visualización.}}
\date{}
\author{{\bf Diego Méndez Medina  420004358}}


\begin{document}

\maketitle

\tableofcontents

\vspace{1cm}
Como proyecto final de la materia Programación Declarativa se realizo
una herramienta ({\it software}) escrita en {\tt HASKELL} que nos permite
trabajar con automátas finitos traductores definidos en archivos
y cambiarlos entre sus tipos (Moore y Mealy). Se visualiza la equivalencia y cuando es
mejor uno sobre el otro.
\begin{multicols}{2}
  
  \section{Antecedentes}

  El automáta finito({\it AF}), con sus múltiples variantes,
  es el modelo aceptador más {\it primitivo}. Pero
  si modificamos, agregando una función, el modelo
  lo convertimos, en no solo una maquína aceptadora,
  sino también en una maquina traductora.

  La definición de la función(de traducción) y sus posibles variaciones
  es lo que nos permite, para {\it AF}, obtener dos traductores\cite{Viso}.
  
  Como experiencía personal y hablando con compañeros
  que cursaron con otrxs profesores, no se suele
  hablar de automátas traductores en los cursos de Automátas
  y Lenguajes Formales que se dan en la Facultad de Ciencias.

  El presente proyecto surgio como idea para familiarizarme
  más con el concepto y también para tener una herramienta
  que nos permita mostrar la traducción entre automátas traductores.  
  \section{Analísis del Problema}
  
  El uso de automátas traductores es más {\it popular} en {\it aplicaciones}
  que el de los automatás de aceptación. Esto por que los primeros
  nos suelen permiter trabajar con toda la cadena, dar más información de esta
  y así {\bf extender} el poder de computo sobre las mismas cadenas.

  Un uso común es en el de compiladores para lexers pero también tienen
  un uso muy conocido en el desarollo de circuitos y encriptación.

  No todo automáta traductor es de aceptación\footnote{A pesar de
  no ser aceptadores para que se lleve acabo la traducción el tipo de cadenas
  se debe de poder aceptar para algún {\it AF}.\cite{Viso}}, el presente se enfoca
  solo en automatás traductores que también son de aceptación. Si bien
  los no aceptadores también tienen aplicaciones para la generación de
  circuitos, lexers y encriptación es necesario tener un inicio.
  
  Como bien ya mencionamos para {\it AF} hay dos tipos de traductores y su
  diferencia radica en la forma en la que la función de respuesta o traducción
  esta definida. Veremos ahora la definición de cada uno.
  
  \subsection{Automata de Moore}

  La respuesta de estos automátas depende {\bf exclusivamente} del estado
  en el que se encuentra el automáta, así es independiente de la entrada.
. Un automáta de Moore, también llamada
  {\it de respuesta asignada por estado}, es una tupla
  $M =(Q,\Sigma, O, \delta, \lambda, q_0, F)$, donde:

  \begin{align*}
    Q\ &: \text{Conjunto de estados}\\
    \Sigma\ &:\  \text{Alfabeto de entrada}\\
    O\ &:\  \text{Alfabeto de salida}\\
    \delta\ &:\  \text{Función de transición}\\
    &:\  Q\times \Sigma \rightarrow Q\\
    \lambda &:\  \text{Función de respuesta}\\
    &:\  Q\times \Sigma\rightarrow O^*\\
    q_0\ &:\  \text{Estado inicial}\\
    F\ &:\  \text{Conjunto de estados Finales}
  \end{align*}
  
  \subsection{Automata de Mealy}
  
  Por otro lado en estos automátas dan su respuesta durante el proceso de la cadena.

  También llamados automátas con respuesta asignada durante la transición es un tupla
  $M =(Q,\Sigma, O, \delta, \lambda, q_0, F)$, donde:

    \begin{align*}
    Q\ &: \text{Conjunto de estados}\\
    \Sigma\ &:\  \text{Alfabeto de entrada}\\
    O\ &:\  \text{Alfabeto de salida}\\
    \delta\ &:\  \text{Función de transición}\\
    &:\  Q\times \Sigma \rightarrow Q\\
    \lambda &:\  \text{Función de respuesta}\\
    &:\  Q\rightarrow O^*\\
    q_0\ &:\  \text{Estado inicial}\\
    F\ &:\  \text{Conjunto de estados Finales}
    \end{align*}

    Como podemos ver son muy similares pues son una extensión de los {\it AF}
    salvo por como lo hemos dicho la función de traducción o respuesta. Existe
    una relación entre estos automátas y es que es posible hablar de la equivalencía
    entre traductores. Decimos que un automáta de Moore y uno de Mealy son equivalentes
    cuando podemos simular la conducta del otro modelo.
    
  \section{Implementación / Solución}
  
  En {\tt HASKELL} es común trabajar con automátas definiendo la función
  de transición(delta), así ya incluimos los estados y el alfabeto y
  por último damos la función de aceptación, donde en realidad solo
  estamos determinando quienes son finales. Con ese acercamiento,
  si bien eficaz, se pierde la noción conjuntista de los automátas como
  tuplas.
  
  Lo que buscamos con el desarrollo del proyecto no es simplemente
  generar automátas en un archivo con extensión {\it hs} y transformarlo,
  sino crear una herramienta qie nos permita leer cualquier tipo de automáta
  traductor y que mediante un formato especifico (archivo txt) trabajar
  con este:\  aceptando cadenas del automáta traductor recibido,
  traduciendo cadenas en este, pasar del tipo automáta traductor recibido
  al otro tipo y hacer las mismas operaciones con este, por último también
  guardar el automáta generado en un archivo para su posterior consulta.

  {\it Lo siguiente es desarrollar el formato para guardar automátas en un
    archivo para que nuestra herramienta lo lea y dar un panorama de la
    implementación de nuestra herramienta.}
  
  \subsection{Definiendo el formato}

  Vamos a trabajar con archivos de texto plano, donde se busca esten las siguientes
  palabras reservadas:

  \begin{align*}
    &\text{Moore} & &\text{Mealy}\\
    &\text{Finales} & &\text{Estados}\\
    &\text{Entrada} & &\text{Salida}\\
    &\text{Transición} & &\text{Respuesta}\\
    &\text{Inicial}
  \end{align*}

  Cada archivo que represente un traductor debe tener todos de los anteriores salvo que
  solo debe escojer entre {\it Moore} o {\it Mealy}, dependiendo del tipo que sea. Para cada
  una de las palabras reservadas(salvo el tipo) se sigue un signo de igual que describe
  el conjunto.
  \begin{itemize}
  \item Estados: Los estados son cadenas y estan separados por coma
  \item Entrada: Es el alfabeto de entrada son caracteres separados por ','
  \item Salida: Es el alfabeto de entrada son caracteres separados por ','
  \item Finales: Similar a Estados.
  \item Transición: Son de la forma Estado``->''Estado, estan separados por ','
  \item Respuesta: Función de traducción y depende del tipo.
    \begin{itemize}
    \item Para Moore: Son de la forma: Estado->Salida
    \item Para Mealy: Son de la forma: Estado->Entrada->Salida
    \end{itemize}
  \item Inicial: Un único estado.
  \end{itemize}
  
  Al finalizar la descripción de cada conjunto se finaliza con un '.'. Se permite que haya
  saltos de línea entre descripciones para no tener líneas muy largas. Es decir es valido tanto:

  $$ Respuesta: q0->0->0, q0->1->1, q4->0->0.$$

  como:

  \begin{align*}
    Respuesta: &q0->0->0, q0->1->1, \\
    & q4->0->0.    
  \end{align*}

  Así mostramos a continuación la representación de dos automátas:

  
  \begin{align*}
    Moore&.\\
    Estados &= A,B,C,D.\\
    Entrada &= 0,1.\\
    Salida &= 0,1,2,3,4.\\
    Transicion &= A->0->A, A->1->B, \\
    &B->0->B, B->1->C,\\
    &C->0->C, C->1->D, \\
    &D->0->D, D->1->A.\\
    Respuesta &= A->0, B->1, C->2, D->3.\\
    Inicial &= A.\\
    Finales &= A,B,C,D.\\  
  \end{align*}

  Automáta de Mealy que dada una cadena de bits de multiplos de cuatro
  traduce la cadena de tal forma que para cuada cuatro bits que lea
  los reproduce y genera un quinto bit de tal forma que el número de unos es impar. 
  \begin{align*}
    Mealy&.\\
    Inicial    &= q0.\\
    Finales    &= q0.\\
    Estados    &= q0, q1, q2, q3, q4, q5, q6.\\
    Entrada    &= 0, 1.\\
    Salida     &= 0, 1.\\
    Respuesta &= q0->0->0, q0->1->1,\\
    &q4->0->0, q4->1->1,\\
    &q1->0->0, q1->1->1, \\
    &q5->0->0, q5->1->1,\\
    &q2->0->0, q2->1->1, \\
    &q6->0->01, q6->1->10,\\
    &q3->0->00, q3->1->11.\\
    Transicion  &= q0->0->q4, q0->1->q1,\\
    & q4->0->q5, q4->1->q2,\\
    &q1->0->q2, q1->1->q5,\\
    &q5->0->q6, q5->1->q3,\\
    &q2->0->q3, q2->1->q6,\\
    &q6->0->q0, q6->1->q0,\\
    &q3->0->q0, q3->1->q0.\\  
  \end{align*}
  
  Para las palabras reservadas no hay distinción entre mayuscuas y minusculas, es decir
  {\it Moore, moore, MoOre, MOre} y demás combinación son válidas. No confundir con
  el lado derecho de la igualdad pues ahi si hay diferencia entre el estado {\it A} y {\it a},
  sucede lo mismo con alfabetos. Tambíen no importa el uso de espacios, saltos de línea
  ni tabulaciónes siempre y cuando esten las ocho palabras del automaáta finalizadas con
  su respectivo punto. Por último no importa el orden en el que aparezcan. Como se vio en
  los ejemplos. Estos automátas y otros ejemplos se encuentran en la carpeta {\it automatas/}
  del repositorio.
  
  \subsection{Sobre la Implementación}

  Vimos a cada traudctor como una extensión de los {\it AF}, y es así que definimos los
  automátas de la siguiente manera:

  \lstinputlisting[language=Haskell, firstline=1, lastline=27]{../src/Automata.hs}

  Observamos que en efecto en lo único
  que varian es en la función de respuestas.
  También observamos que tratamos a las funciones
  como tuplas donde los primeros elementos son
  el dominio y el último el regreso.
  
  Así definimos las funciones normales para automátas que nos permiten procesar cadenas:

  Transita sobre el autómata dada la cadena
  \lstinputlisting[language=Haskell, firstline=78, lastline=78]{../src/Automata.hs}

  Acepta el autómata
  \lstinputlisting[language=Haskell, firstline=98, lastline=98]{../src/Automata.hs}

  Su implementación como las funciones auxiliares se encuentran en la carpeta src
  del repositorio.

  \subsection{Traduciendo Traductores}

  
  \section{Conclusión}
  Es por eso que si bien estos traducen no suelen ser efectivos
  para un método de encriptación general, pues como veremos a la hora de traducir necesitan
  un mayor numero de estados
  \section{Revisión Bibliográfica}  
  \begin{thebibliography}{1}
  \bibitem{Viso} Elisa Viso, E. ``Introducción a Autómatas y Lenguajes Formales'',
    UNAM, CDMX, 2015.
  \bibitem{Cinvestav} Guillermo Morales, L.  ``Máquinas secuenciales'',
    CINVESTAV, CDMX, Junio del 21. Recuperado en Junio del 2023 de el siguiente
    \href{https://delta.cs.cinvestav.mx/~gmorales/ta/node48.html}{enlace}.
  \end{thebibliography}  
\end{multicols}
\end{document}
